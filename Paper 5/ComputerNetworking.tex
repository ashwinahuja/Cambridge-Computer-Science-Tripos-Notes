\documentclass{article}
\title{Bioinformatis}
\author{Ashwin Ahuja}
\usepackage{float}
\usepackage{graphicx}
\usepackage{listings}
\usepackage{xcolor}
\usepackage{tabto}
\usepackage{amssymb}
\usepackage[T1]{fontenc}
\newenvironment{definition}{\par\color{blue}}{\par}
\newenvironment{pros}{\par\color[rgb]{0.066, 0.4, 0.129}}{\par}
\newenvironment{cons}{\par\color{red}}{\par}
\newenvironment{example}{\par\color{brown}}{\par}
\usepackage{fancyhdr}
%% Margins
\usepackage{geometry}
\geometry{a4paper, hmargin={2cm,2cm},vmargin={2cm,2cm}}

%% Header/Footer
\pagestyle{fancy} 
\lhead{Ashwin Ahuja}
\chead{Bioinformatics}
\rhead{Part II}
\lfoot{}
\cfoot{\thepage}
\rfoot{}
\renewcommand{\headrulewidth}{1.0pt}
\renewcommand{\footrulewidth}{1.0pt}

\usepackage[export]{adjustbox}
\usepackage{caption}
\captionsetup{justification   = raggedright,
	singlelinecheck = false}

\lstset{
	basicstyle=\ttfamily,
	columns=fullflexible,
	breaklines=true,
	postbreak=\raisebox{0ex}[0ex][0ex]{\color{red}$\hookrightarrow$\space}
}
\usepackage{listings}
\lstset{
	escapeinside={(*}{*)}
}



\begin{document}
\begin{titlepage}
\begin{center}
			\vspace*{1cm}
			
			\Huge
			\textbf{Bioinformatics}
			
			\vspace{0.5cm}
			\LARGE
			University of Cambridge
			
			\vspace{1.5cm}
			
			\textbf{Ashwin Ahuja}
			
			\vfill
			
			Computer Science Tripos \\
			Part II
			
			\vspace{5cm}
			
			January 2020
			
\end{center}
\end{titlepage}

\tableofcontents
\pagebreak

\section{Purpose}
Main purpose is to help doctors and help to understand the biology and computing with DNA and other biological molecules. Can use DNA as storage information. Today, through \textit{Oxford Nanopore} and \textit{Bento Lab} sequencing has got much cheaper and quicker. Idea of \textbf{Garage Genomics} where it often only takes an hour to get a portion of the DNA.

\subsection{Definitions}
\begin{definition}
\begin{itemize}
\item DNA made of four-letter alphabet: Adenine, Thymine, Cytosine and Guanine
\item RNA is the same with Uracil instead of Thymine. 
\item \textbf{Amino Acid}: Made of three base pairs of DNA
\item Genome: organism's genetic material - for humans = 46 strings (chromosomes) with length 3 x 10e9
\end{itemize}
\end{definition} 



\section{Alignment}

\section{Folding}

\section{Trees and Phylogeny}

\section{Parsimony}

\section{Neighbour Joining}

\section{Genome Sequencing}

\section{Clustering}

\section{Genome Assembly and Pattern Matching}

\section{Hidden Markov Models}

\end{document}